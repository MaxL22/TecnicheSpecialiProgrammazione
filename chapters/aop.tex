\section{Aspect Oriented Programming}

%s2,3
Spesso (ma non sempre), il codice relativo a un certo aspetto (funzione?) è diviso in più punti, ad esempio: all'interno di tomcat
* il codice per l'xml parsing è tutto in un posto solo, modulare e riutilizzabile
* il codice per il logging è sparso in tutta la code base, non è modulare, non è riutilizzabile
%s4
Avere codice sparpagliato per la stessa funzionalit porta a codice ridondante, lo stesso frammento di codice in più posti diversi, inoltre complica la struttura del codice, rendendolo meno comprensibili (e.g., è molto più facile capire del codice senza un sacco di println in mezzo).
Questo porta a codice non chiaro e difficile da mantenere.

%s5
...
Lo scopo di un concern è chiaro, ha una struttura facilmente definibile, ma l'implementazione porta a mischiare un po' le cose (crosscutting?).

L'obiettivo è riuscire a catturare la struttura dei crosscutting concerns ...

%s6
Gli aspetti sono due cose:
* concerns che fanno crosscutting
* un costrutto del linguaggio di programmazione che abilita unità modulari per i crosscutting concerns
La prima è solamente a design level, la seconda permette anche l'implementazione.

AspectJ è ...

%s7?