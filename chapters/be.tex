%slide 11, poi slide 10? Sono invertite
\section{Bytecode Engineering}

Con bytecode engineering intendiamo, preso il contenuto di una classe (di un file \texttt{.class}), riuscire a modificare il codice, prima dell'istanziazione degli oggetti. In questo modo le modifiche vengono applicate a load time, senza rischiare istanze non-modificate "in giro" (non vuol dire che a runtime sia impossibile fare modifiche, ma è più difficile).
% C'è un video di loro che cambiano un gioco a runtime per questa cosa

Alcuni tool per la bytecode instrumentation:
\begin{itemize}
    \item \href{https://asm.ow2.io/}{\texttt{ASM}}

    \item \href{https://commons.apache.org/proper/commons-bcel/}{\texttt{BCEL - Bytecode Engineering Library}}

    \item \href{https://www.javassist.org/}{\texttt{Javassist - Java programming Assistant}}

    \item \href{https://github.com/soot-oss/soot}{\texttt{Soot}}/\href{https://github.com/soot-oss/SootUp}{\texttt{SootUp}}
\end{itemize}

%s4

L'unica cosa che non possiamo andare a modificare è il main, in quanto non si possono modificare le funzioni/metodi già in esecuzione, perchè?
...

Uno dei problemi di Java è che la classe deve passare attraverso il verifier, il quale verifica che la classe sia conforme a? Bisogna in qualche modo?
...

%s5

%s6

%s7
... quando compilato, viene aggiunto un costrutture di default che non fa altro che creare un oggetto

%s8

...

%s12
Attributes sono le annotazioni, per poi andare a ritrovarle?

%s13
Abbiamo la definizione in C del ClassFile, in quanto la JVM è scritta in C.
...
Le versioni (soprattuto la major) servono a stabilire le versioni di Java in cui funziona il file.
...
constant pool contiene tutto ciò che è statico: nomi, ...
...
access flags funzionano come i permessi linux, dai permessi esce un numero

%s14
Bytecode Instrumentation
Motivi: ...

%up to s15

%Change slide, 11?

Non Standard MOPs: Javassist

%intro s2

%s3
la fase di loading è "dirottata" a un translator, che restituisce la classe modificata.

%Ct sta per Concrete

...
%s7
Per modificare le classi, ci sono alcune limitazioni (anche se non è sempre chiaro il perché). Nello specifico, si può:
...

%s8
Si può un po' ignorare l'esistenza del bytecode, nonstante stiamo effettivametne modificando il bytecode stesso.
...
Voglio aggiornare il software senza doverlo riscrivere, vogliamo trasformare applicazioni che usano la vecchia interfaccia con la nuova, senza ricompilare.
Per le modifiche banali, note ad alto livello, serve qualcuno che scrive il bytecode da sostituire: il compilatore lo sa fare

%s9

L'adapter non fa altro ...

Chiede al compilatore di scrivere il bytecode e lo sostituisce tramite reflection.

%to end
